\documentclass[a4paper,12pt]{article}

\usepackage[top = 2cm, bottom = 2cm, left = 2cm, right = 2cm]{geometry} 

\usepackage[T1]{fontenc}
\usepackage[utf8]{inputenc}
\usepackage[spanish]{babel}
\usepackage{multirow}
\usepackage{booktabs} 
\usepackage{graphicx} 
\usepackage{setspace}
\setlength{\parindent}{0in}
\usepackage{float}
\usepackage{fancyhdr}
\usepackage{enumitem}
\usepackage{tikz-timing}
\usepackage{amsmath,amssymb,amsfonts}
\usepackage{hyperref} % Support for hyperlinks
\usepackage{wrapfig}
\usepackage{xcolor}

\tikzset{
timing/z/.style={color=red},
timing/l/.style={color=red},
timing/h/.style={color=red},
timing/slope=0,
timing/name/.append style={yshift=1.5mm}
}

\pagestyle{fancy} 
\fancyhf{} .
\lhead{\footnotesize Introducción al Software de Entretenimiento: Laboratorio 2}

\rhead{\footnotesize 20 de septiembre del 2020}

\cfoot{\footnotesize \thepage} 

\begin{document}

\thispagestyle{empty}

\begin{tabular}{p{15.5cm}} 
\large Universidad Nacional de San Agustín \\ 
\large Facultad de Ingeniería de Producción y Servicios \\
\large Escuela Profesional de Ingeniería de Sistemas \\
{\LARGE \bf Introducción al Software de Entretenimiento} \\
\vspace{1mm}
Alumno: Mestas Escarcena, Carlos Alberto \\
Grupo: B \\
\hline \\
\end{tabular} 

\begin{center} 
	{\LARGE \bf Laboratorio 2}
	\vspace{2mm}
\end{center}  

El desarrollo de este informe se puede encontrar en el repositorio de \textcolor{blue}{
    \href{https://github.com/CarlosMestas/TC_A_9_Carlos_Mestas}{GitHub}}.

\clearpage
\newpage







\section{IPSec}

\setlength{\parindent}{1em} IPsec (Internet Protocol security) está compuesto por varios protocolos de seguridad, diseñado para garantizar que los datos enviados por una red IP sean invisibles e inaccesibles para terceros, permitiendo a dos o más equipo comunicarse de forma segura. Este cifrado se utiliza para garantizar la confidencialidad y para la autenticación.
\\
\\
Funciona como otros protocolos de seguridad en la capa de aplicación de la comunicación de red. Una de sus ventajas es que IPsec, opera a nivel de red en lugar a nivel de aplicación, puede encriptar un paquete completo de IP, mediante dos mecanismos:

\begin{itemize}
    \item \textbf{Encabezado de autenticación (AH):} Mediante la colocación de una firma digital en cada paquete que se envía, protegiendo su red y sus datos a la interferencia de terceros, estos datos no pueden modificarse sin detección y permite también la verificación de identidad entre los dos extremos de la conexión.
    \item \textbf{Carga de Seguridad Encapsulada (ESP):} Por un lado AH evita adulteración de un paquete, ESP garantiza que la información del paquete este encriptada y no pueda leerse.
\end{itemize}

Detalles técnicos:

\begin{itemize}
    \item Compatible con Windows 7+, Windows Server 2008, routers Cisco, macOS y dispositivos iOS.
    \item Admite versiones compatibles para Linux y otros sistemas operativos.
    \item El protocolo principal es el Intercambio de claves de Internet (IKE).
    \item Utiliza la Asociación de seguridad de Internet y el Protocolo de Administración de Claves (ISAKMP) como se define en IETF RFC 2408 para implementar la negociación del servicio VPN.
\end{itemize}

Observando los detalles técnicos previos, cualquier computadora que cumpla con estos destalles, podrá ser configurable para IPSec.
\\
\\
Ya que IPSec se ejecuta en la capa de red, los cambios solo se deben realizar en el sistema operativo, no en las aplicaciones individuales. IPSec también es invisible en funcionamiento y el uso de AH y ESP garantiza un gran alto nivel de seguridad y privacidad. Por otro lado es más complejo que otros protocolos de seguridad, por ellos es más complicado de configurar, IPSec requiere también una clave pública segura, ya que si roban la llave o esta mal administrada puede experimentar problemas y finalmente para la transmisión de paquetes de datos pequeños, puede ser una forma ineficiente de encriptar datos.

\clearpage
\newpage

\section{Parámetros que miden la calidad de servicio sobre Internet}

\begin{itemize}
    \item \textbf{Ancho de banda: }Es la cantidad de datos que se pueden transferir en una red, entre dos puntos en un tiempo específico, normalmente se suele medir en bits por segundo (bps).
    \item \textbf{Tasas de errores: }Es el número de veces que una transmisión de bits recibidos ha sido modificada por interferencia, ruido, errores de sincronización, fluctuación de fase o distorsión. Es una relación de rendimiento de red para transmisiones digitales a través de enlaces de datos de radio, Ethernet o redes de datos de fibra óptica.
    \item \textbf{Jitter: } Es la magnitud de variación del retardo, es un factor crítico en las aplicaciones de tiempo real, si es mayor a la variación de retardo permitida, mayor es el retardo para entregar los datos y más grande es el tamaño de la memoria temporal necesaria en los receptores, en el caso de las teleconferencias, pueden requerir un límite superior razonable para la variación del retardo.
    \item \textbf{Pérdidas de paquetes: }Al navegar por Internet, es posible que se pierda uno o varios paquetes en la transferencia, al no llegar a la dirección de destino, uno puede experimentar servicios lentos, interrupciones de la red e inclusive hasta la pérdida total de conectividad de la red.
    \item \textbf{Interrupciones: }Momentos en los cuales el servicio se detiene, generando de esta manera molestar con respecto al servicio, se detiene tanto la descarga y subida de datos.
    \item \textbf{Retraso en la transmisión: }Es el tiempo que se requiere para empujar todos los bits de un paquete al medio de la transmisión en uso.
    \item \textbf{Disponibilidad: }Es proporcionar acceso ininterrumpido al internet, superando fallas de hardware y software.
    \item \textbf{Tiempo de respuesta de los servicios: } Es el tiempo de respuesta del servidor.
    \item \textbf{Ratio señal a ruido: }Es utilizada para comparar el nivel de una señal deseado con el nivel de ruido de fondo, relación entre la potencia de la señal y la potencia del ruido.
    \item \textbf{Diafonía: }Es la interacción o acoplamiento entre señales cercanas, es cuando parte de las señales parece en otro lado, se produce por un acoplamiento capacitivo o inductivo entre dos cables próximos, pudiendo acoplar señales no deseadas en sistemas de transmisión de datos o telefonía.
    \item \textbf{Eco: }Es el tiempo que transcurre entre la transmisión de una señal y su regreso al receptor. Es causado por las señales reflejadas por el equipo del extremo distante que regresa al oído hablante. Puede llegar a ser un problema significativo cuando el retardo del viaje llega a ser mayor a 50 milisegundos, a medida que el eco se incremente, los sistemas de paquetes van a necesitar usar controles como la cancelación del eco.
    \item \textbf{Rendimiento: }Es la tasa media de transferencia de datos exitosa, es el valor útil para comprender la velocidad habitual de una conexión, se suele medir en bytes por segundo y se puede comparar con el ancho de banda efectivo y el máximo teórico para que de esa manera se pueda determinar si esta bien o mal una conexión.
    
\end{itemize}




 



\end{document}
